\documentclass{article}
\usepackage[utf8]{inputenc}
\usepackage{program}
\usepackage{listings}

\author{Groupe CATisfaction}
\title{La création de CATisfaction : Le site de rencontre pour nos amis félins}

\begin{document}
\pagenumbering{gobble}
\maketitle
\newpage

\pagenumbering{gobble}
\tableofcontents
\newpage

\pagenumbering{arabic}

\section{Introduction}

\subsection{Le groupe}
Le groupe "CATisfaction" est composé des étudiants suivants : 
\begin{itemize}
\item Thomas "Psiich0" BERATTO 
\item Corentin "Tuareg" LAFOND
\item Quentin "Docteur" DELAVELLE
\item Adrien "Fontaine" VIZIER
\end{itemize}

\subsection{Le projet}
On connaît tous \textbf{Tinder}, \textbf{Meetic} ou les autres sites de rencontres. A l'heure où le numérique se développe et où les sorties se raréfient, ces sites sont de plus en plus plébiscités par une population de tous les âges. \textbf{CATisfaction} s'inspire de ce succès pour l'adapter à un tout autre domaine. En effet, \textbf{CATisfaction} reconditionne le concept de match et de recherche de partenaire pour l'appliquer au monde félin. Le site s'adresse à toutes ces personnes qui aiment trop leur compagnon pour envisager de le stériliser, et qui souhaitent, parce qu'elles n'en peuvent plus des miaulements de leur femelle en chaleur, ou bien parce qu'elles recherchent le partenaire idéal pour conserver la lignée de leur chat de race.
\newline
\newline
Premier mondial à se lancer dans le secteur, \textbf{CATisfaction} répond à une recherche qui ne faisait jusqu'alors que de bouches à oreilles ou bien entre éleveurs associés pour l'élargir au niveau mondial grâce à Internet.

\newpage
\section{Réalisation du site}

\subsection{Architecture générale}
Afin d'éviter les répétitions, il a été décidé de créer des pages \textbf{top} et \textbf{bottom} qui sont réutilisés systématiquement car elles comportent la bannière et la signature du site. Néanmoins, il a fallu diviser la première citée puisque celle-ci contenait alors l'intégralité du \textbf{head} et empêcher d'implémenter des fonctions en Javascript.
\newline
\newline
Pour ce qui est du reste, on prend évidemment soin de distinguer deux menus différents selon que l'utilisateur soit connectés ou pas. De plus, grâce à un système de ROOTPATH, le titre de chaque page est généré automatiquement.

\subsection{Système de matching}
Les partenaires potentiels d'un chat de l'utilisateur sont affichés dynamiquement dans un \textbf{tableau}. Les tableaux des \textit{targets} de chaque chats sont stockés sous forme de chaînes de caractères dans un tableau. Ils sont ainsi créés en PHP puis insuflés dans le script JvS à l'aide de la fonction \textbf{json\_encode()}.
\newline
Pour sélectionner les chats correspàondant au désireux, on récupères d'abord dans la base de donnée tout les chats du bon sexe et de la bonne race si le désireux tient à la pureté du sang. A chaque chat est ensuite attribué un score, augmentant d'un certain \textbf{coefficient} pour chacun des nombreux critères demandés par le désireux qu'il respecte. Ces \textbf{coefficients} sont aujourd'hui le fruit d'une simple approche intuitive, mais l'étude des prochains clients via des calculs statistiques permettra de trouver des sytèmes plus efficaces que ces simples \textbf{coefficients}. Les chats sont ensuite triés par scores décroissants, et seuls les 5 premiers sont enregistrés. De même on pourra garder un plus grand nombre de profil si cela s'avère plus efficace. 

\subsection{Design}
Le designe de notre site abuse du phénomène de toxoplasmose, autrement dit, cette attirence quasi-surnaturelle de la population humaine pour les chats (même en vidéo). De ce fait, nous avons parcemé les pages d'images de nos félins préférés, ce qui devrait sans doute attirer autant de millions d'utilisateurs que les millions de vues des vidéos de \textbf{Youtube}.
\newline
\newline
Pour accentuer cela, nous avons également transformé le curseur de la souris afin qu'il prenne la forme d'une patoune !!.
\newpage

\section{Pour aller plus loin}
Pour aller plus loin on peut imaginer des \textbf{critères inversés}, c'est-à-dire des critères non désirés en plus des critères désirés.
\newline
On aurait aussi aimé mettre en place un système de communication entre les utilisateurs \textbf{interne au site}, mais cela demande du temps que nous n'avons pas encore.
\newline
Il est possible que face à un très grand nombre de chats sur le serveur, certaines requêtes puissent prendre du temps. En étudiant le code pourrions parvenir à \textbf{limiter ces délais}, mais aujourd'hui cela est loin d'être un problème.
\newline
Actuellement le site ne permet pas les rencontres entre chats \textbf{du même sexe}. L'homosexualité n'étant pas inconnue au monde félin, la possibilité de rencontres mono-genre pourra facilement être ajouté à l'avenir, les \textbf{variables} nécessaires étant déjà en place.
\end{document}
